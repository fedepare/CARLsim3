\hypertarget{ch1_getting_started_ch1s1_preinstallation}{}\section{1.\+1 Pre-\/\+Installation}\label{ch1_getting_started_ch1s1_preinstallation}
\begin{DoxyAuthor}{Author}
Kristofor D. Carlson
\end{DoxyAuthor}
C\+A\+R\+Lsim runs on both generic x86 C\+P\+Us and off-\/the-\/shelf N\+V\+I\+D\+IA G\+P\+Us. Full support of all C\+A\+R\+Lsim features requires that the N\+V\+I\+D\+IA C\+U\+DA parallel computing platform be installed. It is now also possible to compile C\+A\+R\+Lsim without G\+PU support (see \hyperlink{ch1_getting_started_ch1s2s1s3_cpuonly}{1.\+2.\+1.\+3 Installing C\+A\+R\+Lsim Without G\+PU Support}).

C\+A\+R\+Lsim requires G\+P\+Us with a compute capability of 2.\+0 or higher. To find the compute capability of your device please refer to the \href{http://en.wikipedia.org/wiki/CUDA}{\tt {\bfseries C\+U\+DA article on Wikipedia}}.

C\+A\+R\+Lsim also requires C\+U\+DA Toolkit 5.\+0 or higher. For platform-\/specific C\+U\+DA installation instructions, please navigate to the \href{https://developer.nvidia.com/cuda-zone}{\tt {\bfseries N\+V\+I\+D\+IA C\+U\+DA Zone}}.

The rest of the chapter assumes you have successfully installed C\+U\+DA on appropriate hardware.\hypertarget{ch1_getting_started_ch1s1s1_os}{}\subsection{1.\+1.\+1 Supported Operating Systems}\label{ch1_getting_started_ch1s1s1_os}
C\+A\+R\+Lsim has been tested on the following platforms\+:
\begin{DoxyItemize}
\item Windows 7
\item Windows 8
\item Ubuntu 12.\+04
\item Ubuntu 12.\+10
\item Ubuntu 13.\+04
\item Ubuntu 13.\+10
\item Ubuntu 14.\+04
\item Arch Linux
\item Cent\+OS 6
\item Open\+S\+U\+SE 13.\+1
\item Mac OS X
\end{DoxyItemize}\hypertarget{ch1_getting_started_ch1s1s2_os}{}\subsection{1.\+1.\+2 Source Directory Layout}\label{ch1_getting_started_ch1s1s2_os}
Below is the directory layout of the C\+A\+R\+Lsim source code. All source code of the core library is contained in the directory {\ttfamily carlsim}. The sub-\/directories are key components to the C\+A\+R\+Lsim simulation library.

The {\ttfamily doc} directory contains doxygen-\/related source files in {\ttfamily source} and the pre-\/compiled H\+T\+ML version of the documentation in {\ttfamily html}.

The {\ttfamily projects} directory contains a template for writing your first C\+A\+R\+Lsim program. Users will start here when they begin writing their first program.

The {\ttfamily tools} directory contains a number of C\+A\+R\+Lsim plugins that may be useful to users such as parameter tuning frameworks, M\+A\+T\+L\+AB scripts, spike generators, and tools for visual stimuli.


\begin{DoxyCode}
├── carlsim                       # CARLsim source code directory
│   ├── connection\_monitor          # Utility to record synaptic data
│   ├── group\_monitor               # Utility to record neuron group data
│   ├── \textcolor{keyword}{interface                   }# CARLsim interface (public user interface)
│   ├── kernel                      # CARLsim core functionality
│   ├── server                      # Utility to implement real-time server functionality
│   ├── spike\_monitor               # Utility to record neuron spike data
│   └── test                        # Google test regression suite tests
├── doc                           # CARLsim documentation generation directory
│   ├── html                        # Generated documentation in html
│   └── source                      # Documentation source code
├── projects                      # User projects directory
│   └── hello\_world                 # Project template for new users
└── tools                         # CARLsim tools that are not built-in
    ├── ecj\_pti                     # Automated parameter-tuning interface using ECJ
    ├── eo\_pti                      # Automated parameter-tuning interface using EO (deprecated)
    ├── offline\_analysis\_toolbox    # Collection of MATLAB scripts for data analysis
    ├── simple\_weight\_tuner         # Simple weight parameter-tuning tool
    ├── spike\_generators            # Collection of input spike generation tools
    └── visual\_stimulus             # Collection of MATLAB/CARLsim tools for visual stimuli
\end{DoxyCode}


\begin{DoxySince}{Since}
v3.\+0
\end{DoxySince}
\hypertarget{ch1_getting_started_ch1s2_installation}{}\section{1.\+2 Installation}\label{ch1_getting_started_ch1s2_installation}
\begin{DoxyAuthor}{Author}
Kristofor D. Carlson 

Ting-\/\+Shuo Chou 

Michael Beyeler
\end{DoxyAuthor}
To install C\+A\+R\+Lsim, first download and unzip the zip file from the \href{http://www.socsci.uci.edu/~jkrichma/CARLsim/index.html}{\tt C\+A\+RL website}. For installation instructions on Linux and Mac OS X platforms, please refer to \hyperlink{ch1_getting_started_ch1s2s1_linux}{1.\+2.\+1 Linux / Mac OS X} below. For installation instructions on Windows platforms, please refer to \hyperlink{ch1_getting_started_ch1s2s2_windows}{1.\+2.\+2 Windows} below.\hypertarget{ch1_getting_started_ch1s2s1_linux}{}\subsection{1.\+2.\+1 Linux / Mac O\+S X}\label{ch1_getting_started_ch1s2s1_linux}
Instructions for Linux/\+Mac OS X installation assume you are using the Bash shell. Additionally, the G\+NU G\+CC compiler collection and G\+NU Make build environment should be installed. On most platforms, these are already installed by default. After you have unzipped the downloaded C\+A\+R\+Lsim files, you next have to set installation-\/specific environment variables such as information about G\+PU devices, C\+U\+DA Toolkit version, and the desired installation location.\hypertarget{ch1_getting_started_ch1s2s1s1_environment_variables}{}\subsubsection{1.\+2.\+1.\+1 Environment Variables}\label{ch1_getting_started_ch1s2s1s1_environment_variables}
The easiest way to set all relevant environment variables is to add the following lines to your {\ttfamily $\sim$/.bashrc} file\+: 
\begin{DoxyCode}
\textcolor{preprocessor}{# CARLsim mandatory variables}
export CARLSIM\_LIB\_DIR=/opt/CARL/CARLsim
export CUDA\_INSTALL\_PATH=/usr/local/cuda
export CARLSIM\_CUDAVER=6
export CUDA\_MAJOR\_NUM=3
export CUDA\_MINOR\_NUM=5

\textcolor{preprocessor}{# CARLsim optional variables}
export CARLSIM\_FASTMATH=0
export CARLSIM\_CUOPTLEVEL=3
export CPU\_ONLY=0
\end{DoxyCode}


The desired installation location of the C\+A\+R\+Lsim library is specified with the {\ttfamily C\+A\+R\+L\+S\+I\+M\+\_\+\+L\+I\+B\+\_\+\+D\+IR} variable. The major and minor compute capability numbers of your C\+U\+D\+A-\/capable G\+PU device must be specified by setting the {\ttfamily C\+U\+D\+A\+\_\+\+M\+A\+J\+O\+R\+\_\+\+N\+UM} and {\ttfamily C\+U\+D\+A\+\_\+\+M\+I\+N\+O\+R\+\_\+\+N\+UM} variables, respectively. Next {\ttfamily C\+U\+D\+A\+\_\+\+I\+N\+S\+T\+A\+L\+L\+\_\+\+P\+A\+TH} variable must be set. This variable points to where C\+U\+DA is installed. Finally, the C\+U\+DA Toolkit version must be set with the {\ttfamily C\+A\+R\+L\+S\+I\+M\+\_\+\+C\+U\+D\+A\+V\+ER} variable. {\ttfamily C\+A\+R\+L\+S\+I\+M\+\_\+\+F\+A\+S\+T\+M\+A\+TH} (G\+CC fast-\/math flag) and {\ttfamily C\+A\+R\+L\+S\+I\+M\+\_\+\+C\+U\+O\+P\+T\+L\+E\+V\+EL} (optimization level, disable with value 0, enable with values 1-\/3) are optional settings. Setting {\ttfamily C\+P\+U\+\_\+\+O\+N\+LY} to 1 will allow you to compile C\+A\+R\+Lsim without any G\+PU support.

To make sure these settings go into effect, you can either type\+: 
\begin{DoxyCode}
$ source ~/.bashrc
\end{DoxyCode}
 or close the shell and open another one.

An alternative way to set the required environment variable is to edit the {\ttfamily user.\+mk} file found in the C\+A\+R\+Lsim root directory. This can be helpful in case multiple users share the same C\+A\+R\+Lsim installation and want to use global configuration settings. The \textquotesingle{}=?\textquotesingle{} sign in {\ttfamily user.\+mk} indicates the value the variable will be assigned if it is not already defined in the {\ttfamily $\sim$/.bashrc}. For example, the following line from {\ttfamily user.\+mk} would assign value 5 to environment variable {\ttfamily C\+A\+R\+L\+S\+I\+M\+\_\+\+C\+U\+D\+A\+V\+ER} if the variable does not already exist\+: 
\begin{DoxyCode}
CARLSIM\_CUDAVER ?= 5
\end{DoxyCode}
\hypertarget{ch1_getting_started_ch1s2s1s2_cuda_version}{}\subsubsection{1.\+2.\+1.\+2 Finding C\+U\+D\+A Toolkit Version and Compute Capability}\label{ch1_getting_started_ch1s2s1s2_cuda_version}
The C\+U\+DA Toolkit version can be found via\+: 
\begin{DoxyCode}
$ nvcc --version
\end{DoxyCode}
 You need only input the major number of the toolkit version (e.\+g. 6 for 6.\+5).

The compute capability of the G\+PU device can be found by compiling the {\ttfamily device\+Query} sample in the directory {\ttfamily 1\+\_\+\+Utilities} of the C\+U\+DA Toolkit. 
\begin{DoxyCode}
\textcolor{preprocessor}{# copy NVIDIA Toolkit to home directory}
$ cd /usr/local/cuda/bin
$ ./cuda-install-samples-6.5.sh ~
$ cd ~/NVIDIA\_CUDA-6.5\_Samples/1\_Utilities/deviceQuery

\textcolor{preprocessor}{# compile and run deviceQuery}
$ make
$ ./deviceQuery
\end{DoxyCode}
 For C\+U\+DA Toolkits other than version 6.\+5, the paths above need to be changed accordingly.\hypertarget{ch1_getting_started_ch1s2s1s3_cpuonly}{}\subsubsection{1.\+2.\+1.\+3 Installing C\+A\+R\+Lsim Without G\+P\+U Support}\label{ch1_getting_started_ch1s2s1s3_cpuonly}
To install C\+A\+R\+Lsim without G\+PU support, the environment variable {\ttfamily C\+P\+U\+\_\+\+O\+N\+LY} in {\ttfamily user.\+mk} needs to be set to 1\+: 
\begin{DoxyCode}
CPU\_ONLY ?= 1
\end{DoxyCode}
 Alternatively, the variable can be added to {\ttfamily $\sim$/.bashrc}.

In this case, the N\+V\+I\+D\+IA C\+U\+DA toolkit is not required. But, obviously you will not be able to run C\+A\+R\+Lsim in \+::\+G\+P\+U\+\_\+\+M\+O\+DE.\hypertarget{ch1_getting_started_ch1s2s1s4_compiling}{}\subsubsection{1.\+2.\+1.\+4 Compiling the C\+A\+R\+Lsim Library}\label{ch1_getting_started_ch1s2s1s4_compiling}
After the environment variables have been set, C\+A\+R\+Lsim can be compiled and installed via\+: 
\begin{DoxyCode}
$ make
$ sudo make install
\end{DoxyCode}


This is will install the C\+A\+R\+Lsim library in the location pointed to by {\ttfamily C\+A\+R\+L\+S\+I\+M\+\_\+\+L\+I\+B\+\_\+\+D\+IR} (see \hyperlink{ch1_getting_started_ch1s2s1s1_environment_variables}{1.\+2.\+1.\+1 Environment Variables} above).

C\+A\+R\+Lsim comes with an optional automated parameter tuning framework. For more information about how to install the framework please see ch10\+\_\+ecj. Additionally, C\+A\+R\+Lsim now comes with a regression suite that uses \href{https://code.google.com/p/googletest/}{\tt Google Test}. For more information on how to use the regression suite, please see ch11\+\_\+regression\+\_\+suite.

\begin{DoxySince}{Since}
v3.\+0
\end{DoxySince}
\hypertarget{ch1_getting_started_ch1s2s2_windows}{}\subsection{1.\+2.\+2 Windows}\label{ch1_getting_started_ch1s2s2_windows}
C\+A\+R\+Lsim provides solution files for Microsoft Visual Studio (VS) 2012 and C\+U\+DA 5.\+5. The solution file is called {\ttfamily C\+A\+R\+Lsim.\+sln} and is located in the C\+A\+R\+Lsim root directory. In addition, every project, tutorial, and the regression suite have their own .vcxproj projects file in the appropriate directory.

Before building the solution, {\ttfamily Configuration} should be set to {\ttfamily x64}. {\ttfamily Release} should be selected for project executables, and {\ttfamily Debug} should be selected for compiling the regression suite.

VS 2012 will then generate all executables (.exe) and the static library (.lib) via \char`\"{}\+Build Solution\char`\"{}.

Newer VS versions will automatically upgrade the solution file ({\ttfamily C\+A\+R\+Lsim.\+sln}) and all project files ({\ttfamily $\ast$.vcxproj}). For newer C\+U\+DA Toolkit versions, the strings \char`\"{}\+C\+U\+D\+A 5.\+5.\+props\char`\"{} and \char`\"{}\+C\+U\+D\+A 5.\+5.\+targets\char`\"{} that are present in every .vcxproj file have to be manually updated to reflect the right C\+U\+DA Toolkit version number.

C\+A\+R\+Lsim comes with an optional automated parameter tuning framework. For more information about how to install the framework please see ch10\+\_\+ecj. Additionally, C\+A\+R\+Lsim now comes with a regression suite that uses \href{https://code.google.com/p/googletest/}{\tt Google Test}. For more information on how to use the regression suite, please see ch11\+\_\+regression\+\_\+suite.\hypertarget{ch1_getting_started_ch1s3_project_workflow}{}\section{1.\+3 Project Workflow}\label{ch1_getting_started_ch1s3_project_workflow}
\begin{DoxyAuthor}{Author}
Kristofor D. Carlson 

Ting-\/\+Shuo Chou
\end{DoxyAuthor}
A sample \char`\"{}\+Hello World\char`\"{} project is provided in the {\ttfamily projects/hello\+\_\+world} directory. The project includes a single source file {\ttfamily main\+\_\+hello\+\_\+world.\+cpp} that creates a network with two groups, connected with random weights, and can be used as a skeleton to create new projects.

Any output files created by the simulation will be automatically placed in the {\ttfamily results/} directory.

All M\+A\+T\+L\+AB scripts should be placed in the {\ttfamily scripts/} directory. This directory already contains two M\+A\+T\+L\+AB scripts to aid in using the O\+AT (see ch9\+\_\+matlab\+\_\+oat). The script {\ttfamily init\+O\+A\+T.\+m} adds the O\+AT directory to the M\+A\+T\+L\+AB path, whereas {\ttfamily demo\+O\+A\+T.\+m} will open a Network\+Monitor to visualize network activity. Note that for {\ttfamily demo\+O\+A\+T.\+m} to work, the executable must be run first (see \hyperlink{ch1_getting_started_ch1s3s1s1_linux_hello_world_compile}{1.\+3.\+1.\+1 Compiling and Running the \char`\"{}\+Hello World\char`\"{} Project in Linux / Mac OS X} and \hyperlink{ch1_getting_started_ch1s3s2s1_win_hello_world_compile}{1.\+3.\+2.\+1 Compiling and Running the \char`\"{}\+Hello World\char`\"{} Project in Windows} below). In order to run the O\+AT, open M\+A\+T\+L\+AB, change to {\ttfamily projects/hello\+\_\+world/scripts/}, then type\+: 
\begin{DoxyCode}
>> initOAT     % adds OAT relative path to MATLAB paths
>> demoOAT     % opens a NetworkMonitor on the simulation file
\end{DoxyCode}
\hypertarget{ch1_getting_started_ch1s3s1_linux_project_workflow}{}\subsection{1.\+3.\+1 Linux / Mac O\+S X}\label{ch1_getting_started_ch1s3s1_linux_project_workflow}
\hypertarget{ch1_getting_started_ch1s3s1s1_linux_hello_world_compile}{}\subsubsection{1.\+3.\+1.\+1 Compiling and Running the \char`\"{}\+Hello World\char`\"{} Project in Linux / Mac O\+S X}\label{ch1_getting_started_ch1s3s1s1_linux_hello_world_compile}
The \char`\"{}\+Hello World\char`\"{} project comes with its own Makefile that compiles the file {\ttfamily main\+\_\+hello\+\_\+world.\+cpp} and links it with the C\+A\+R\+Lsim library. The project can be compiled and run with the following set of commands\+: 
\begin{DoxyCode}
$ cd projects/hello\_world
$ make
$ ./hello\_world
\end{DoxyCode}


Any output files created by the simulation will be automatically placed in the {\ttfamily results/} directory. This may include any spike files created by Spike\+Monitor, a debug log file, and a network structure file.

All local objects and executables can be deleted via\+: 
\begin{DoxyCode}
$ make clean
\end{DoxyCode}


All output files, including local objects, executables, and files in the {\ttfamily results/} directory can be deleted via\+: 
\begin{DoxyCode}
$ make distclean
\end{DoxyCode}


\begin{DoxyWarning}{Warning}
When {\ttfamily make distclean} is called, all data files in the results directory will be deleted!
\end{DoxyWarning}
\hypertarget{ch1_getting_started_ch1s3s1s2_linux_create_new}{}\subsubsection{1.\+3.\+1.\+2 Creating a New Project in Linux / Mac O\+S X}\label{ch1_getting_started_ch1s3s1s2_linux_create_new}
The easiest way to create a new project in Linux/\+Mac OS X is to make a copy of the {\ttfamily projects/hello\+\_\+world/} directory and all its corresponding subdirectories, rename the directory accordingly, and place it alongside {\ttfamily hello\+\_\+world/} in the {\ttfamily projects/} directory. Then only minimal changes to the Makefile must be made in order for the project to compile correctly.

The Makefile provided in the directory was made so that users only have to modify a small portion of the file to build a custom project. Below is the modifiable portion of the Makefile\+: 
\begin{DoxyCode}
\textcolor{preprocessor}{# Makefile for building project program from the CARLsim library}

\textcolor{preprocessor}{# NOTE: if you are compiling your code in a directory different from}
\textcolor{preprocessor}{# examples/<example\_name> or projects/<project\_name> then you need to either}
\textcolor{preprocessor}{# move the configured user.mk file to this directory or set the path to}
\textcolor{preprocessor}{# where CARLsim can find the user.mk.}
USER\_MK\_PATH = ../../
include $(USER\_MK\_PATH)user.mk

project := hello\_world
output := *.dot *.dat *.log *.csv
\end{DoxyCode}
 The {\ttfamily U\+S\+E\+R\+\_\+\+M\+K\+\_\+\+P\+A\+TH} variable points to the {\ttfamily user.\+mk} file in the C\+A\+R\+Lsim root directory. This file is needed because it contains all necessary compilation and linking flags. If the {\ttfamily user.\+mk} is moved to a different location, the {\ttfamily U\+S\+E\+R\+\_\+\+M\+K\+\_\+\+P\+A\+TH} needs to be updated accordingly.

The name of the project can be changed via variable {\ttfamily project}. Whatever string is assigned here will influence the name of the Makefile target as well as the name of the C++ source file. For example, setting {\ttfamily project} to \char`\"{}hello\+\_\+world\char`\"{} will assume that a source file {\ttfamily main\+\_\+hello\+\_\+world.\+cpp} exists, and will create an executable called {\ttfamily hello\+\_\+world}.

Finally, files and/or file extensions to be deleted with the {\ttfamily make clean} and {\ttfamily make distclean} commands can be edited by changing the {\ttfamily output} variable.

\begin{DoxyNote}{Note}
The C++ source file must be named {\ttfamily main\+\_\+\{project name\}.cpp} for the Makefile to compile correctly, where {\ttfamily \{project\+\_\+name\}} is the string assigned to the {\ttfamily project} variable in the Makefile.
\end{DoxyNote}
\hypertarget{ch1_getting_started_ch1s3s2_windows_project_workflow}{}\subsection{1.\+3.\+2 Windows}\label{ch1_getting_started_ch1s3s2_windows_project_workflow}
\hypertarget{ch1_getting_started_ch1s3s2s1_win_hello_world_compile}{}\subsubsection{1.\+3.\+2.\+1 Compiling and Running the \char`\"{}\+Hello World\char`\"{} Project in Windows}\label{ch1_getting_started_ch1s3s2s1_win_hello_world_compile}
The \char`\"{}\+Hello World\char`\"{} project comes with its own {\ttfamily .vcxproj} project file that has already been added to the {\ttfamily C\+A\+R\+Lsim.\+sln} solution file. Thus the project can be built simply by opening the {\ttfamily C\+A\+R\+Lsim.\+sln} solution file in VS, right-\/clicking the project directory and choosing \char`\"{}\+Build project\char`\"{}.\hypertarget{ch1_getting_started_ch1s3s2s1_win_create_new}{}\subsubsection{1.\+3.\+2.\+2 Creating a New Project in Windows}\label{ch1_getting_started_ch1s3s2s1_win_create_new}
The easiest way to create new project in Windows is to make a copy of the directory {\ttfamily projects/hello\+\_\+world} and all its corresponding subdirectories, to rename the directory accordingly, and to place it alongside {\ttfamily hello\+\_\+world/} in the {\ttfamily projects/} directory. Then only minimal changes to the project and solution file need to be made in order for the project to compile correctly.

First, the project file in the new project directory needs to be named according to the new project name\+: {\ttfamily \{project name\}.vcxproj}. The C++ source file should be renamed for consistency\+: {\ttfamily main\+\_\+\{project name\}.cpp}.

Second, the projects file needs to be added to the {\ttfamily C\+A\+R\+Lsim.\+sln} solution file.

Then the new project is ready to be built, rebuilt, or cleaned directly through VS.

\begin{DoxySince}{Since}
v3.\+0
\end{DoxySince}
\hypertarget{ch1_getting_started_ch1s4_uninstallation}{}\section{1.\+4 Uninstallation}\label{ch1_getting_started_ch1s4_uninstallation}
\begin{DoxyAuthor}{Author}
Michael Beyeler
\end{DoxyAuthor}
\hypertarget{ch1_getting_started_ch1s4s1_linux}{}\subsection{1.\+4.\+1 Linux / Mac O\+S X}\label{ch1_getting_started_ch1s4s1_linux}
To uninstall C\+A\+R\+Lsim on a Unix platform, open a terminal, navigate to the C\+A\+R\+Lsim root directory, and type\+: 
\begin{DoxyCode}
$ sudo make uninstall
\end{DoxyCode}


This will remove the directory pointed to by the environment variable {\ttfamily C\+A\+R\+L\+S\+I\+M\+\_\+\+L\+I\+B\+\_\+\+D\+IR}. By default, this variable points to the location {\ttfamily \char`\"{}/opt/\+C\+A\+R\+L/\+C\+A\+R\+Lsim\char`\"{}}.

\begin{DoxyNote}{Note}
Any environments that have been added to {\ttfamily $\sim$/.bashrc} must be removed manually. 
\end{DoxyNote}
\begin{DoxyAttention}{Attention}
Before uninstalling, make sure that the environment variable {\ttfamily C\+A\+R\+L\+S\+I\+M\+\_\+\+L\+I\+B\+\_\+\+D\+IR} is properly set.
\end{DoxyAttention}
\hypertarget{ch1_getting_started_ch1s4s2_windows}{}\subsection{1.\+4.\+2 Windows}\label{ch1_getting_started_ch1s4s2_windows}
On Windows, simply move all downloaded and unzipped C\+A\+R\+Lsim files to the recycle bin. 